%%%%%%%%%%%%%%%%%%%%%%%%%%%%%%%%%%%%%%%%%
% Journal Article
% LaTeX Template
% Version 2.0 (February 7, 2023)
%
% This template originates from:
% https://www.LaTeXTemplates.com
%
% Author:
% Vel (vel@latextemplates.com)
%
% License:
% CC BY-NC-SA 4.0 (https://creativecommons.org/licenses/by-nc-sa/4.0/)
%
% NOTE: The bibliography needs to be compiled using the biber engine.
%
%%%%%%%%%%%%%%%%%%%%%%%%%%%%%%%%%%%%%%%%%

%----------------------------------------------------------------------------------------
%	PACKAGES AND OTHER DOCUMENT CONFIGURATIONS
%----------------------------------------------------------------------------------------

\documentclass[
	a4paper, % Paper size, use either a4paper or letterpaper
	10pt, % Default font size, can also use 11pt or 12pt, although this is not recommended
	unnumberedsections, % Comment to enable section numbering
	twoside, % Two side traditional mode where headers and footers change between odd and even pages, comment this option to make them fixed
]{LTJournalArticle}

\addbibresource{sample.bib} % BibLaTeX bibliography file

%\runninghead{Shortened Running Article Title} % A shortened article title to appear in the running head, leave this command empty for no running head

\footertext{\textit{Space technologies minor project} (2023)} % Text to appear in the footer, leave this command empty for no footer text

\setcounter{page}{1} % The page number of the first page, set this to a higher number if the article is to be part of an issue or larger work

%----------------------------------------------------------------------------------------
%	TITLE SECTION
%----------------------------------------------------------------------------------------

\title{X-Ray Observation of Venus\\ with the INTEGRAL Telescope} % Article title, use manual lines breaks (\\) to beautify the layout

% Authors are listed in a comma-separated list with superscript numbers indicating affiliations
% \thanks{} is used for any text that should be placed in a footnote on the first page, such as the corresponding author's email, journal acceptance dates, a copyright/license notice, keywords, etc
\author{%
	Kent Barbey\textsuperscript{1} \\ \textbf{Supervisor:} Volodymyr Savchenko\textsuperscript{1}
}

% Affiliations are output in the \date{} command
\date{\footnotesize\textsuperscript{\textbf{1}}LASTRO, School of Physics, Ecole Polytechnique Fédérale de Lausanne, Switzerland}

% Full-width abstract
\renewcommand{\maketitlehookd}{%
	\begin{abstract}
            On April 22 and 24, 2022, Venus was serendipitously observed with the JEM-X detector of the \textit{INTEGRAL} space telescope. The observation performed yielded

            \textbf{Keywords: }
	\end{abstract}
}

%----------------------------------------------------------------------------------------

\begin{document}

\maketitle % Output the title section
\tableofcontents
%----------------------------------------------------------------------------------------
%	ARTICLE CONTENTS
%----------------------------------------------------------------------------------------

\section{Introduction}
    \subsection{INTEGRAL telescope}
    \subsection{ODA-API}
    \subsection{Sunpy}
    \subsection{X-Ray emission of planets}

%------------------------------------------------

\section{Observation, data analysis and methods}

    \subsection{Data selection}
        %selection criteria of scw: sun activity, elongation and presence of scw
    \begin{figure}[H]
        \centering
        \includegraphics[width = 12cm]{report/Figures/methods/Positional_astronomy.png}
        \caption{Caption}
        \label{elongation}
    \end{figure}
    
    \begin{figure}[H]
        \centering
        \includegraphics[width = 12cm]{report/Figures/methods/Venus_position.png}
        \caption{Caption}
        \label{venus_pos}
    \end{figure}
    
    %put here the different scw used
    \paragraph{22.04.2022 data}

    The April 22 data consists of...
    
    \textbf{Fig.} \ref{22_flux_map} shows ...
    
        \begin{figure}[H]
        \centering
        \begin{subfigure}{.3\textwidth}
            \includegraphics[width=\textwidth]{report/Figures/methods/2204/20_map.png}
        \end{subfigure}%
        \hspace{1em}-
        \begin{subfigure}{.3\textwidth}
            \centering
            \includegraphics[width=\textwidth]{report/Figures/methods/2204/21_map.png}
        \end{subfigure}
        \hspace{1em}-
        \begin{subfigure}{.3\textwidth}
            \centering
            \includegraphics[width=\textwidth]{report/Figures/methods/2204/22_map.png}
        \end{subfigure}
        \hspace{1em}-
        \begin{subfigure}{.3\textwidth}
            \centering
            \includegraphics[width=\textwidth]{report/Figures/methods/2204/23_map.png}
        \end{subfigure}
        \hspace{1em}-
        \begin{subfigure}{.3\textwidth}
            \centering
            \includegraphics[width=\textwidth]{report/Figures/methods/2204/24_map.png}
        \end{subfigure}
        \hspace{1em}-
        \begin{subfigure}{.3\textwidth}
            \centering
            \includegraphics[width=\textwidth]{report/Figures/methods/2204/25_map.png}
        \end{subfigure}
        \hspace{1em}-
        \begin{subfigure}{.3\textwidth}
            \centering
            \includegraphics[width=\textwidth]{report/Figures/methods/2204/26_map.png}
        \end{subfigure}
        \hspace{1em}-
        \begin{subfigure}{.3\textwidth}
            \centering
            \includegraphics[width=\textwidth]{report/Figures/methods/2204/27_map.png}
        \end{subfigure}
        \hspace{1em}-
        \begin{subfigure}{.3\textwidth}
            \centering
            \includegraphics[width=\textwidth]{report/Figures/methods/2204/31_map.png}
        \end{subfigure}
        \hspace{1em}-
        \begin{subfigure}{.3\textwidth}
            \centering
            \includegraphics[width=\textwidth]{report/Figures/methods/2204/32_map.png}
        \end{subfigure}
        \hspace{1em}-
        \begin{subfigure}{.3\textwidth}
            \centering
            \includegraphics[width=\textwidth]{report/Figures/methods/2204/33_map.png}
        \end{subfigure}
        \caption{}
        \label{22_flux_map}
        \end{figure}

        \begin{figure}[H]
        \centering
        \includegraphics[width = 12cm]{report/Figures/methods/2204/oda_2204.png}
        \caption{Caption}
        \label{22_mosaic}
        \end{figure}
    
    \paragraph{24.04.2022 data}
    The April 24 data consists of ...

    \begin{figure}[H]
        \centering
        \begin{subfigure}{.45\textwidth}
            \includegraphics[width=\textwidth]{report/Figures/methods/2404/24_map.png}
        \end{subfigure}%
        \hspace{1em}-
        \begin{subfigure}{.45\textwidth}
            \centering
            \includegraphics[width=\textwidth]{report/Figures/methods/2404/25_map.png}
        \end{subfigure}
        \hspace{1em}-
        \begin{subfigure}{.45\textwidth}
            \centering
            \includegraphics[width=\textwidth]{report/Figures/methods/2404/26_map.png}
        \end{subfigure}
        \hspace{1em}-
        \begin{subfigure}{.45\textwidth}
            \centering
            \includegraphics[width=\textwidth]{report/Figures/methods/2404/27_map.png}
        \end{subfigure}
        \caption{}
        \label{24_map}
    \end{figure}

        \begin{figure}[H]
        \centering
        \includegraphics[width = 12cm]{report/Figures/methods/2404/oda_2404.png}
        \caption{Caption}
        \label{24_mosaic}
        \end{figure}

    
    \subsection{Models and assumptions}

        \begin{figure}[H]
        \centering
        \begin{subfigure}{.45\textwidth}
            \includegraphics[width=\textwidth]{report/Figures/models/model_psf_const.png}
        \end{subfigure}%
        \hspace{1em}-
        \begin{subfigure}{.45\textwidth}
            \centering
            \includegraphics[width=\textwidth]{report/Figures/models/model_psf_const_3d.png}
        \end{subfigure}
        \caption{}
        \label{model_psf_const}
        \end{figure}

        \begin{figure}[H]
        \centering
        \begin{subfigure}{.45\textwidth}
            \includegraphics[width=\textwidth]{report/Figures/models/model_psf_notconst.png}
        \end{subfigure}%
        \hspace{1em}-
        \begin{subfigure}{.45\textwidth}
            \centering
            \includegraphics[width=\textwidth]{report/Figures/models/model_psf_notconst_3d.png}
        \end{subfigure}
        \caption{}
        \label{model_psf_notconst}
        \end{figure}

        \begin{figure}[H]
        \centering
        \includegraphics[width = \textwidth]{report/Figures/models/2204/threshold_determination_notconst.png}
        \caption{Caption}
        \label{threshold}
        \end{figure}

    
    %--------------------------------------------------
    
    \subsection{Solar events}
    
    \begin{figure}[H]
        \centering
        \includegraphics[width = 12cm]{report/Figures/methods/coordinates.png}
        \caption{Caption}
        \label{coordinates}
    \end{figure}

    \begin{figure}[H]
        \centering
        \includegraphics[width = \textwidth]{report/Figures/methods/GOES_total.png}
        \caption{Caption}
        \label{goes_tot}
    \end{figure}
    

%------------------------------------------------

\section{Results}
    \subsection{Fluxes results}

    \paragraph{22.04.2022 results}
    ???

        \begin{figure}[H]
        \centering
        \begin{subfigure}{\textwidth}
            \includegraphics[width=\textwidth]{report/Figures/results/lc_2204.png}
        \end{subfigure}%
        \hspace{1em}-
        \begin{subfigure}{\textwidth}
            \centering
            \includegraphics[width=\textwidth]{report/Figures/results/lc_2204_psf_notconst.png}
        \end{subfigure}
        \hspace{1em}-
        \begin{subfigure}{\textwidth}
            \centering
            \includegraphics[width=\textwidth]{report/Figures/results/lc_2204_psf_const.png}
        \end{subfigure}
        \caption{}
        \label{22_lc}
        \end{figure}
        

    \paragraph{24.04.2022 results}
    ???
        \begin{figure}[H]
        \centering
        \begin{subfigure}{\textwidth}
            \includegraphics[width=\textwidth]{report/Figures/results/lc_2404.png}
        \end{subfigure}%
        \hspace{1em}
        \begin{subfigure}{\textwidth}
            \centering
            \includegraphics[width=\textwidth]{report/Figures/results/lc_2404_notconst.png}
        \end{subfigure}
        \hspace{1em}
        \begin{subfigure}{\textwidth}
            \centering
            \includegraphics[width=\textwidth]{report/Figures/results/lc_2404_psf_const.png}
        \end{subfigure}
        \caption{}
        \label{24_lc}
        \end{figure}

    \begin{figure}[H]
        \centering
        \includegraphics[width = 12cm]{report/Figures/results/spectra_comp.png}
        \caption{Caption}
        \label{comp_spec}
    \end{figure}
    
    \subsection{Concordance with solar flux variations}

    \begin{figure}[H]
        \centering
        \begin{subfigure}{\textwidth}
            \includegraphics[width=\textwidth]{report/Figures/results/GOES_22.png}
        \end{subfigure}%
        \hspace{1em}
        \begin{subfigure}{\textwidth}
            \centering
            \includegraphics[width=\textwidth]{report/Figures/results/norm_22.png}
        \end{subfigure}
        \caption{}
        \label{goes22}
    \end{figure}

    \begin{figure}[H]
        \centering
        \begin{subfigure}{\textwidth}
            \includegraphics[width=\textwidth]{report/Figures/results/GOES_24.png}
        \end{subfigure}%
        \hspace{1em}-
        \begin{subfigure}{\textwidth}
            \centering
            \includegraphics[width=\textwidth]{report/Figures/results/norm_24.png}
        \end{subfigure}
        \caption{Caption}
        \label{goes_24}
    \end{figure}
    
    \subsection{Solar events locations}
    blablabla

        \begin{figure}[H]
        \centering
        \begin{subfigure}{.47\textwidth}
            \includegraphics[width=\textwidth]{report/Figures/results/cme_loc.png}
        \end{subfigure}%
        \hspace{1em}
        \begin{subfigure}{.47\textwidth}
            \centering
            \includegraphics[width=\textwidth]{report/Figures/results/fl_loc.png}
        \end{subfigure}
        \caption{}
        \label{locator}
        \end{figure}

    fsldfj
    \begin{figure}[H]
        \centering
        \includegraphics[width = 12cm]{report/Figures/results/aia_Mclass_2204.png}
        \caption{Caption}
        \label{Mclass_flare}
    \end{figure}

    

%------------------------------------------------

\section{Discussion}
%talk about fluorescence maybe
%talk about the next steps, what's to do: check the abundances or check with models. Cite some articles(check the word document). See how many photons would be expected. 
% would be cool to put limits on processes: charge-exchange, fluorescence of heavy elements etc...
%say that the second method is probably the most correct one
%talk about the significance tests.
% compare with literature

%say that both assumptions have argument for and against them but not constant is probably better because of variability.

The fluxes retrieved from Venus' position all show similar averages. The PSF modelling is however the better way to retrieve Venus' flux as it is a somewhat punctual source with very limited extent in the image given its apparent size and the detector's resolution. The question still arises though whether the non-constant or constant model is the best one. Since the solar flux is the main influencer of the X-ray emission of the planet and since it varies on the time scale of minutes, the non-constant model would appear to be the best one especially in high activity periods of the Sun.

Compared to \cite{Dennerl2002DiscoveryChandra}, the order of magnitude of the flux found is the same. However, the expected behaviour would be a decrease in the average flux and not an increase as \textbf{Fig.} \ref{comp_spec} suggests. The \textit{Dennerl} paper uses only fluorescence as a model and this could be accounted for the discrepancy. The next step of the analysis reported here would be to put thresholds on the different possible emission processes such as charge exchange which was modelled for the Venus atmosphere in \cite{Gombosi1981THEABSORPTION}.

The KS tests performed on the vertical distributions of the fluxes all reject the idea of a direct correlation between the solar flux variability and Venus' although it is less clear for the 24.04.2022 window(4.8\%). This is the same conclusion that arised from \cite{Dennerl2002DiscoveryChandra} and the same explanation can be made: Venus was seeing another portion of the Sun and the solar flux variability detected were localised and didn't necessarily impact Venus. The number of points is low though and increasing the statistics would be the next step.

The events are all directed towards Earth and shows the FoV limitation of LASCO and GOES to Earth directed events only. Venus sees a rotated side of the Sun by about $\sim 45$°. But the number of events (54 CMEs and 130 flares $\geq$ C) detected in the time interval from the 19.04.2022 to the 24.04.2022 is high. There are therefore no reason to think that the Sun's activity wasn't as high on the Venus' directed side. The fluxes retrieved here should only be used as general indicators of the Sun's activity not as general events directly impacting Venus although the events, in particular the most powerful ones, often have spatial extents of 180° and therefore impacting both planets at the same time. Locating the direction of events not directed towards Earth requires either an X-ray spacecraft to be orbiting the Sun at the desired position or using other methods such as triangulation \cite{Liu2010GeometricAU} with the STEREO spacecrafts in heliocentric orbit. This was out of the scope of this project and not explored. The \cite{Dennerl2002DiscoveryChandra} study had the same trouble as they used the GOES missions solar flux data.

%------------------------------------------------

\section{Conclusion}

This statement requires citation \autocite{Futaana2017SolarAtmosphere}. This statement requires multiple citations \autocite{Futaana2017SolarAtmosphere, Afshari2016X-RAYINGMAGNETOTAIL}. This statement contains an in-text citation, for directly referring to a citation like so: \textcite{Futaana2017SolarAtmosphere}.

\section{Links}

This is a clickable URL link: \href{https://www.latextemplates.com}{LaTeX Templates}. This is a clickable email link: \href{mailto:vel@latextemplates.com}{vel@latextemplates.com}. This is a clickable monospaced URL link: \url{https://www.LaTeXTemplates.com}.

%------------------------------------------------
\section{Annex A}
This section contains all subsidiary sky plots and relevant data and models used to obtain the results.

\subsection{22.04 sky plots}

        \begin{figure}[H]
        \centering
        \begin{subfigure}{.3\textwidth}
            \includegraphics[width=\textwidth]{report/Figures/methods/2204/20_map.png}
        \end{subfigure}%
        \hspace{1em}-
        \begin{subfigure}{.3\textwidth}
            \centering
            \includegraphics[width=\textwidth]{report/Figures/methods/2204/21_map.png}
        \end{subfigure}
        \hspace{1em}-
        \begin{subfigure}{.3\textwidth}
            \centering
            \includegraphics[width=\textwidth]{report/Figures/methods/2204/22_map.png}
        \end{subfigure}
        \hspace{1em}-
        \begin{subfigure}{.3\textwidth}
            \centering
            \includegraphics[width=\textwidth]{report/Figures/methods/2204/23_map.png}
        \end{subfigure}
        \hspace{1em}-
        \begin{subfigure}{.3\textwidth}
            \centering
            \includegraphics[width=\textwidth]{report/Figures/methods/2204/24_map.png}
        \end{subfigure}
        \hspace{1em}-
        \begin{subfigure}{.3\textwidth}
            \centering
            \includegraphics[width=\textwidth]{report/Figures/methods/2204/25_map.png}
        \end{subfigure}
        \hspace{1em}-
        \begin{subfigure}{.3\textwidth}
            \centering
            \includegraphics[width=\textwidth]{report/Figures/methods/2204/26_map.png}
        \end{subfigure}
        \hspace{1em}-
        \begin{subfigure}{.3\textwidth}
            \centering
            \includegraphics[width=\textwidth]{report/Figures/methods/2204/27_map.png}
        \end{subfigure}
        \hspace{1em}-
        \begin{subfigure}{.3\textwidth}
            \centering
            \includegraphics[width=\textwidth]{report/Figures/methods/2204/31_map.png}
        \end{subfigure}
        \hspace{1em}-
        \begin{subfigure}{.3\textwidth}
            \centering
            \includegraphics[width=\textwidth]{report/Figures/methods/2204/32_map.png}
        \end{subfigure}
        \hspace{1em}-
        \begin{subfigure}{.3\textwidth}
            \centering
            \includegraphics[width=\textwidth]{report/Figures/methods/2204/33_map.png}
        \end{subfigure}
        \caption{Flux map from every SCW where Venus is present on the 22.04.2022. The start and end positions of Venus are circled in white and red respectively. Another bright source is also to verify the correct localisation approach.}
        \label{22_flux_map}
        \end{figure}

\subsection{24.04 sky plots}

\begin{figure}[H]
        \centering
        \begin{subfigure}{.45\textwidth}
            \includegraphics[width=\textwidth]{report/Figures/methods/2404/24_map.png}
        \end{subfigure}%
        \hspace{1em}-
        \begin{subfigure}{.45\textwidth}
            \centering
            \includegraphics[width=\textwidth]{report/Figures/methods/2404/25_map.png}
        \end{subfigure}
        \hspace{1em}-
        \begin{subfigure}{.45\textwidth}
            \centering
            \includegraphics[width=\textwidth]{report/Figures/methods/2404/26_map.png}
        \end{subfigure}
        \hspace{1em}-
        \begin{subfigure}{.45\textwidth}
            \centering
            \includegraphics[width=\textwidth]{report/Figures/methods/2404/27_map.png}
        \end{subfigure}
        \caption{Flux map from every SCW where Venus is present on the 24.04.2022. The start and end positions of Venus are circled in white and red respectively. Another bright source is also to verify the correct localisation approach.}
        \label{24_map}
    \end{figure}

The 2D gaussian models fitted to obtain the different fluxes of \textbf{Fig.} \ref{22_lc} and \ref{24_lc} are the following.

\subsection{Non constant flux assumption fits}
This section shows the different fits and the subsequent model for the non constant flux assumption described in \textbf{Sec. Observation, data analysis and methods}.
\paragraph{22.04 data}
These correspond to the 22.04.2022 data window.
    \begin{figure}[H]
    \centering
    \begin{subfigure}{.47\textwidth}
        \includegraphics[width=\textwidth]{report/Figures/models/2204/20_psf_notconst.png}
    \end{subfigure}%
    \hspace{1em}-
    \begin{subfigure}{.47\textwidth}
        \centering
        \includegraphics[width=\textwidth]{report/Figures/models/2204/21_psf_notconst.png}
    \end{subfigure}
    \begin{subfigure}{.47\textwidth}
        \centering
        \includegraphics[width=\textwidth]{report/Figures/models/2204/22_psf_notconst.png}
    \end{subfigure}
    \hspace{1em}-
    \begin{subfigure}{.47\textwidth}
        \includegraphics[width=\textwidth]{report/Figures/models/2204/23_psf_notconst.png}
    \end{subfigure}%
    \hspace{1em}-
    \begin{subfigure}{.47\textwidth}
        \centering
        \includegraphics[width=\textwidth]{report/Figures/models/2204/24_psf_notconst.png}
    \end{subfigure}
    \begin{subfigure}{.47\textwidth}
        \centering
        \includegraphics[width=\textwidth]{report/Figures/models/2204/25_psf_notconst.png}
    \end{subfigure}
    \hspace{1em}-
    \begin{subfigure}{.47\textwidth}
        \includegraphics[width=\textwidth]{report/Figures/models/2204/26_psf_notconst.png}
    \end{subfigure}%
    \hspace{1em}-
    \begin{subfigure}{.47\textwidth}
        \centering
        \includegraphics[width=\textwidth]{report/Figures/models/2204/27_psf_notconst.png}
    \end{subfigure}
    \begin{subfigure}{.47\textwidth}
        \centering
        \includegraphics[width=\textwidth]{report/Figures/models/2204/31_psf_notconst.png}
    \end{subfigure}
    \hspace{1em}-
    \begin{subfigure}{.47\textwidth}
        \includegraphics[width=\textwidth]{report/Figures/models/2204/32_psf_notconst.png}
    \end{subfigure}%
    \hspace{1em}-
    \begin{subfigure}{.47\textwidth}
        \centering
        \includegraphics[width=\textwidth]{report/Figures/models/2204/33_psf_notconst.png}
    \end{subfigure}
    \caption{}
    \label{model_const_22}
    \end{figure}
\paragraph{24.04 data}
These correspond to the 24.04.2022 data window.

    \begin{figure}[H]
    \centering
    \begin{subfigure}{.47\textwidth}
        \includegraphics[width=\textwidth]{report/Figures/models/2404/24_psf_notconst.png}
    \end{subfigure}%
    \hspace{1em}-
    \begin{subfigure}{.47\textwidth}
        \centering
        \includegraphics[width=\textwidth]{report/Figures/models/2404/25_psf_notconst.png}
    \end{subfigure}
    \begin{subfigure}{.47\textwidth}
        \centering
        \includegraphics[width=\textwidth]{report/Figures/models/2404/26_psf_notconst.png}
    \end{subfigure}
    \hspace{1em}-
    \begin{subfigure}{.47\textwidth}
        \includegraphics[width=\textwidth]{report/Figures/models/2404/27_psf_notconst.png}
    \end{subfigure}%
    \caption{}
    \label{model_const_24}
    \end{figure}

\subsection{Constant flux assumption fits}

This section shows the different fits and the subsequent model for the constant flux assumption described in \textbf{Sec. Observation, data analysis and methods}.
\paragraph{22.04 data}
These correspond to the 22.04.2022 data window.
    \begin{figure}[H]
    \centering
    \begin{subfigure}{.47\textwidth}
        \includegraphics[width=\textwidth]{report/Figures/models/2204/20_psf_const.png}
    \end{subfigure}%
    \hspace{1em}-
    \begin{subfigure}{.47\textwidth}
        \centering
        \includegraphics[width=\textwidth]{report/Figures/models/2204/21_psf_const.png}
    \end{subfigure}
    \begin{subfigure}{.47\textwidth}
        \centering
        \includegraphics[width=\textwidth]{report/Figures/models/2204/22_psf_const.png}
    \end{subfigure}
    \hspace{1em}-
    \begin{subfigure}{.47\textwidth}
        \includegraphics[width=\textwidth]{report/Figures/models/2204/23_psf_const.png}
    \end{subfigure}%
    \hspace{1em}-
    \begin{subfigure}{.47\textwidth}
        \centering
        \includegraphics[width=\textwidth]{report/Figures/models/2204/24_psf_const.png}
    \end{subfigure}
    \begin{subfigure}{.47\textwidth}
        \centering
        \includegraphics[width=\textwidth]{report/Figures/models/2204/25_psf_const.png}
    \end{subfigure}
    \hspace{1em}-
    \begin{subfigure}{.47\textwidth}
        \includegraphics[width=\textwidth]{report/Figures/models/2204/26_psf_const.png}
    \end{subfigure}%
    \hspace{1em}-
    \begin{subfigure}{.47\textwidth}
        \centering
        \includegraphics[width=\textwidth]{report/Figures/models/2204/27_psf_const.png}
    \end{subfigure}
    \begin{subfigure}{.47\textwidth}
        \centering
        \includegraphics[width=\textwidth]{report/Figures/models/2204/31_psf_const.png}
    \end{subfigure}
    \hspace{1em}-
    \begin{subfigure}{.47\textwidth}
        \includegraphics[width=\textwidth]{report/Figures/models/2204/32_psf_const.png}
    \end{subfigure}%
    \hspace{1em}-
    \begin{subfigure}{.47\textwidth}
        \centering
        \includegraphics[width=\textwidth]{report/Figures/models/2204/33_psf_const.png}
    \end{subfigure}
    \caption{}
    \label{model_notconst_22}
    \end{figure}
\paragraph{24.04 data}
These correspond to the 24.04.2022 data window.

    \begin{figure}[H]
    \centering
    \begin{subfigure}{.47\textwidth}
        \includegraphics[width=\textwidth]{report/Figures/models/2404/24_psf_const.png}
    \end{subfigure}%
    \hspace{1em}-
    \begin{subfigure}{.47\textwidth}
        \centering
        \includegraphics[width=\textwidth]{report/Figures/models/2404/25_psf_const.png}
    \end{subfigure}
    \begin{subfigure}{.47\textwidth}
        \centering
        \includegraphics[width=\textwidth]{report/Figures/models/2404/26_psf_const.png}
    \end{subfigure}
    \hspace{1em}-
    \begin{subfigure}{.47\textwidth}
        \includegraphics[width=\textwidth]{report/Figures/models/2404/27_psf_const.png}
    \end{subfigure}%
    \caption{}
    \label{model_notconst_24}
    \end{figure}


\section{Annex B}
%talk about packages used here.
The \textit{GitHub} repository corresponding to this project can be accessed here:

\url{https://github.com/kbarbey/EE-589.git}

Apart from the usual \texttt{Python} libraries, the most important packages used included:

\begin{itemize}
    \item \texttt{Sunpy 5.0} to access the Heliophysics Event Knowledgebase (HEK) containing the different flare and CME events.
    \item \texttt{Skyfield 1.45} to compute ephemerides.
    \item \texttt{Astropy 5.2.1}
\end{itemize}

Moreover, the \href{https://ssd.jpl.nasa.gov/horizons/app.html}{\textit{JPL Horizons}} web interface was used to compute the illumination of Venus and its apparent size as seen from the \textit{INTEGRAL} telescope (observer site code: 500@-198).

%----------------------------------------------------------------------------------------
%	 REFERENCES
%----------------------------------------------------------------------------------------
\nocite{*}
\printbibliography % Output the bibliography

%----------------------------------------------------------------------------------------

\end{document}
